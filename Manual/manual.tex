\chapter{User Manual}

\section{Introduction}

\section{Installation}
\subsection{system requirements}
in the design section of this work i put in some system requirements to make this program work on most computers those requirements were:
\begin{itemize}
	\item AMD dual core graphics processor running at 3.0GHz
	\item 4GB DDR3 
	\item Sata III 600GB internal HDD
	\item CD-Rom Drive
	\item motherboard onboard graphics
\end{itemize}
the operating system for the computer doesn't matter
\subsection{Prerequisite Installation}
the application doesn't require many programs to be instatlled to the computer apart from itself and the programs listed below 
%include as many subsubsections as necessary for each piece of required software
\subsubsection{Installing Python}
\begin{enumerate}
\item in your internet browser go to the website\newline
https://www.python.org/downloads/
\item download the latest version of python but do not install it until you have downloaded the version of PyQt that corresponds to the version 
\item once PyQt has downloaded open the python file and run the .exe file inside and then fallow the onscreen instructions
\end{enumerate}
\subsubsection{Installing PyQt}
\begin{enumerate}
\item in your internet browser go to the website\newline
http://www.riverbankcomputing.co.uk/software/pyqt/download
\item make sure the version of PyQt is correct to install on your chosen operating system 

\end{enumerate}
\subsubsection{Etc.}

\subsection{System Installation}

\subsection{Running the System}

\section{Tutorial}

\subsection{Introduction}

\subsection{Assumptions}

\subsection{Tutorial Questions}

%include as many subsubsections as necessary for each question in your list
\subsubsection{Question 1}

\subsubsection{Question 2}

\subsection{Saving}

\subsection{Limitations}

\section{Error Recovery}

%include as many subsections as necessary for each error
\subsection{Error 1}

\subsection{Error 2}

\section{System Recovery}

\subsection{Backing-up Data}

\subsection{Restoring Data}
